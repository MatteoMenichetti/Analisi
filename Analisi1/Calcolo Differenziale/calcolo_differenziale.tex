\section{Calcolo differenziale}\label{sec:calc_diff}
\subsection{Introduzione}
Il calcolo differenziale è un sottocampo del calcolo infinitesimale che studia il cambiamento quantitativo di una funzione. L'obbiettivo primario di studio del calcolo differenziale è la derivazione di una funzione, come il \gls{differenziale} di una funzione, e la sua applicazione.

\paragraph{La derivata può assumere il significato fisico di velocità.} Dati due istanti di tempo $t_0<t_1$, la velocità media nell'intervallo $[t_0,t_1]$ è il rapporto della distanza, misurata con la funzione $s$ e la durata dell'intervallo, ovvero:
\begin{equation*}
    v_m=\frac{s(t_1)-s(t_0)}{t_1-t_0}.
\end{equation*}

Se la velocità è irregolare può essere più accurato calcolare la velocità istantanea all'istante $t_0$, ovvero fare tendere $t_1$ a $t_0$:
\begin{equation*}
    v_1=\lim_{t\rightarrow t_0}\frac{s(t)-s(t_0)}{t-t_0}.
\end{equation*}

\paragraph{La derivata può assumere il significato coefficiente angolare della retta tangente.} Data una retta secante al grafico di $f$ che passa per $x_0$ e $x_1$, questa ha coefficiente angolare pari a
\begin{equation*}
    m=\frac{f(x_1)-f(x_0)}{x_1-x_0}.
\end{equation*}

La retta passante per $x_0$ con coefficiente angolare pari a
\begin{equation*}
    m=\lim_{x\rightarrow x_0}\frac{f(x)-s(t_0)}{x-x_0}.
\end{equation*}
è detta retta tangente al grafico di $f$ nel punto di ascissa $x_0$.

\begin{definition}[Rapporto incrementale]
    Sia $f\colon (a,b)\rightarrow\mathbb R$ e $x_0\in(a,b)$. Il rapporto incrementale di $f$ in $x_0$ è 
    \begin{equation}\label{eq:rapporto_incrementale}
    \frac{\Delta f}{\Delta x}=\frac{f(x_0+h)-f(x_0)}{h}.
    \end{equation}
\end{definition}

\begin{definition}[Derivata]
    Sia $f\colon (a,b)\rightarrow\mathbb R$ e $x_0\in(a,b)$. $f$ è derivabile in $x_0$ se esiste finito il limite
    \begin{equation}\label{eq:limite_rapporto_incrementale}
        f'(x_0)=\lim_{h\rightarrow 0}\frac{f(x_0+h)-f(x_0)}{h}
    \end{equation}
\end{definition}

\begin{definition}
    Se $f$ è derivabile in ogni punto di $(a,b)$ è detta derivabile.
\end{definition}

\begin{theorem}
    Se $f$ è derivabile in $x_0$ allora è continua in $x_0$.
\end{theorem}

Non vale il viceversa:
\begin{example}
    $f(x)=|x|$ non è derivabile in 0 perche' il rapporto incrementale in 0 vale $\frac{|h|}{h}$, quindi
    \begin{equation*}
        \lim_{h\rightarrow 0^+}\frac{|h|}{h}=1\neq -1=\lim_{h\rightarrow 0^-}\frac{|h|}{h}.
    \end{equation*}
    ovvero 0 è un punto angoloso.
\end{example}

\begin{definition}[Punto angoloso]
    Se il limite destro e sinistro del rapporto incrementale per $x\rightarrow x_0$ sono finiti ma diversi, $x_0$ è un punto angoloso.
\end{definition}

\begin{definition}[Punto stazionario]
    Sia $f$ definita su $[a,b]$. Un punto $x_0\in[a,b]$ è stazionario se $f'(x_0)=0$.
\end{definition}

\subsection{Regole di derivazione}

\begin{theorem}
    \begin{equation*}
        \frac{d}{dx}x^n=nx^{n-1}.
    \end{equation*}
\end{theorem}

\begin{theorem}
    Se $f$ e $g$ sono funzioni derivabili in $x$, allora valgono le seguenti regole di derivazione:
    \begin{equation*}
        \frac{d}{dx}(\alpha f+\beta g)(x) = \alpha\frac{df}{dx}(x)+\beta\frac{dg}{dx}(x) \quad\text{\textbf{Linearita'}}
    \end{equation*}
    \begin{equation*}
        \frac{d}{dx}(f\cdot g)(x) = \frac{df}{dx}(x)g(x)+f(x)\frac{dg}{dx}(x)
              \quad\text{\textbf{Regola di Leibnitz}}
    \end{equation*}
    \begin{equation*}
        \frac{d}{dx}\left(\frac{f}{g}\right)(x)=\frac{\frac{df}{dx}(x)g(x) - f(x)\frac{dg}{dx}(x)}{g^2(x)}
    \end{equation*}
\end{theorem}

\begin{example}
    \begin{equation*}
        \begin{matrix}
            \frac{d}{dx}\tan x &=& \frac{d}{dx}\frac{\sin x}{\cos x} &=& \frac{\cos^2x+\sin^2 x}{\cos^2 x} &=& \frac{1}{\cos^2 x}&=& \tan^2 x+1,\\\\
            \frac{d}{dx}\cot x &=& \frac{d}{dx}\frac{\cos x}{\sin x} &=& \frac{-\sin^2x-\cos^2 x}{\sin^2x}&=&-\frac{1}{\sin^2x}&=&-1-\cot^2 x.
        \end{matrix}
    \end{equation*}
\end{example}

\subsection{Derivate di funzioni composte}
\begin{theorem}
    Sia $g$ una funzione funzione derivabile in $x$ e $f$ derivabile in $g(x)$. La funzione $f\circ g$ è derivabile in $x$ e
    \begin{equation*}
        \frac{d}{dx}[f(g(x))]=f'(g(x))\cdot g'(x).
    \end{equation*}
\end{theorem}

\begin{example}
    \begin{equation*}
        \frac{d}{dx}(\sin^2x)=2\sin x\cos x,
    \end{equation*}
    \begin{equation*}
        \frac{d}{dx}(\sin x^2)=2x\cos x^2.
    \end{equation*}
\end{example}

\subsection{Derivate di funzioni inverse}
\begin{theorem}
    Sia $f$ una funzione continua e strettamente monotona in $(a,b)$. Se $f$ è derivabile in $x\in(a,b)$ e $f'(x)\neq 0$, allora $f^{-1}$ è derivabile in $y=f(x)$ e
    \begin{equation*}
        Df^{-1}(y)=\frac{1}{f'(x)}=\frac{1}{f'(f^{-1}(y))}.
    \end{equation*}
\end{theorem}

\subsection{Derivate note}
\begin{equation*}
    \begin{matrix}
        D(\tan x) &=& \frac{1}{\cos^2x} &=& \tan^2x+1\\\\
        D(\sin x) &=& \cos x\\\\
        D(\cos x) &=& -\sin x\\\\
        D(\ln{|x|}) &=& \frac{1}{x}\\\\
        D(\log_b{|x|}) &=& \frac{1}{x\ln b} && \text{per } b>0,\, b\neq 1\\\\
        D(a^x) &=& a^x\ln a && \text{per } a>0\\\\
        D(|x|) &=& \frac{x}{|x|}\\\\
        D(x^n) &=& nx^{n-1}
    \end{matrix}
\end{equation*}

\begin{example}
    La funzione $f(x)=x^3$ è continua e derivabile in $(0,+\infty)$. La funzione $f^{(-1)}(y)=\sqrt[3]{y}$ è derivabile in $(0,+\infty)$ e 
    \begin{equation*}
        Df^{-1}(y)=\frac{1}{3x^2}=\frac{1}{3\sqrt[3]{y^2}}
    \end{equation*}
\end{example}

\begin{example}
    La funzione $f(x)=\tan x$ è continua e derivabile in $\left(-\frac{\pi}{2},\frac{\pi}{2}\right)$. Sia $f^{-1}(y)=\arctan y$,
    \begin{equation*}
        D\arctan y=\frac{1}{1+\tan^2 y}=\frac{1}{1+y^2}.
    \end{equation*}
\end{example}

\subsection{Massimi e minimi relativi}
\begin{definition}[Punto di massimo$\backslash$minimo relativo]
    Sia $f$ definita in $[a,b]$. $x_0\in[a,b]$ è un punto di massimo relativo per $f$ in $[a,b]$ se
    \begin{equation*}
        \exists\delta>0\colon f(x_0)\underset{(\leq)}{\geq} f(x)\;\forall x\in[a,b] \text{ con } |x-x_0|<\delta.
    \end{equation*}
\end{definition}

Un punto di massimo o minimo relativo sono estremi locali. Per ulteriori informazioni per trovare minimi e massimi relativi vedere Osservazione \ref{rem:minimo_relativo} e la Sezione \ref{ssec:estremi_locali_derivate}.

\begin{remark}
    Se $f$ è pari e $f(x)=\max$ allora $f(-x)=\max$. Se $f$ è dispari e $f(x)=\max$ allora $-f(x)=f(-x)=\min$.
\end{remark}

\subsection{Risultati importanti}
\begin{theorem}[Fermat]\label{th:teorema_fermat}
    Sia $f$ definita in $[a,b]$ e sia $\boldsymbol{x_0}\in(a,b)$ un \textbf{punto di massimo (o minimo) relativo}. \textbf{Se} $\boldsymbol f$ \textbf{e' derivabile in} $\boldsymbol{x_0}$, allora $\boldsymbol{f'(x_0)=0}$.
\end{theorem}

\begin{remark}
    è possibile notare quanto segue:
    \begin{itemize}
        \item Non vale se l'estremo non è interno all'intervallo \footnote{$f(x)=3x+1$ in $[1,2]$ ha un massimo relativo in 2 ma $f'(x)=3\neq 0$.}.
        \item Non vale se $f$ non è derivabile \footnote{Vedere $f(x)=|x|$ in 0.}.
        \item Non vale il viceversa \footnote{$f(x)=x^3$ non ha estremo in 0}.
    \end{itemize}
\end{remark}

\begin{theorem}
    Sia $f$ continua in $[a,b]$ e derivabile in $(a,b)$. Se $f(a)=f(b)$ esiste un punto $x_0\in(a,b)$ tale che $f'(x_0)=0$
\end{theorem}

\begin{remark}
    In $x_0$ la tangente è orizzontale.
\end{remark}

\begin{theorem}[Lagrange]
    Sia $f$ continua in $[a,b]$ e derivabile in $(a,b)$. Esiste $x_0\in(a,b)$ tale che
    \begin{equation*}
        f'(x_0)=\frac{f(b)-f(a)}{b-a}
    \end{equation*}
\end{theorem}

\begin{theorem}[Cauchy]
    Siano $f$ e $g$ continue in $[a,b]$ e derivabili in $(a,b)$ con $g'(x)\neq 0$. Esiste $x_0\in(a,b)$ tale che
    \begin{equation*}
        \frac{f'(x)}{g'(x)}=\frac{f(b)-f(a)}{g(b)-g(a)}.
    \end{equation*}
\end{theorem}

\subsubsection{Conseguenze Teorema di Lagrange}

\begin{remark}
    Conseguenza del teorema è che esiste una retta tangente al grafico di $f$ nel punto $(x_0, f(x_0))$ parallela alla retta passante per gli estremi $(a,f(a))$ e $(b, f(b))$ (i coefficienti, ovvero le derivate prime delle rette, sono uguali).
\end{remark}

\begin{theorem}
    Data $f(x)$, continua in $[a,b]$ e derivabile in $(a,b)$, se risulta identicamente $f'(x)=0$ allora $f$ è costante in $[a,b]$.
\end{theorem}

\begin{theorem}
    Date $f(x)$ e $g(x)$, continue in $[a,b]$ e derivabili in $(a,b)$, se $f'(x)=g'(x)\,\forall x\in[a,b]$ allora $f$ e $g$ differiscono per una costante. 
\end{theorem}

\begin{theorem}
    Condizioni necessarie e sufficienti affiche' $f(x)$, continua in $[a,b]$ e derivabile in $(a,b)$, sia crescente (oppure decrescente) in $[a,b]$ sono:
    \begin{itemize}
        \item $f'(x)> 0\quad \forall x\in[a,b]$,
        \item $\nexists I\subseteq A$ tale che $f'(x)=0$. 
    \end{itemize}
\end{theorem}


\subsection{Teorema di de l'Hôpital}
\begin{theorem}[di de l'Hôpital]
    Siano $-\infty\leq a<b\leq +\infty$ e $f,g$ funzioni derivabili in $(a,b)$. Supposto che per $c\in(a,b)$ valga
    \begin{equation}\label{eq:hopital_ipotesi}
        \lim_{x\rightarrow c^+}f(x)=0=\lim_{x\rightarrow c^+}g(x)
    \end{equation}
    e
    \begin{equation*}
        g'(x)\neq 0\quad\forall x\in(a,b)\backslash c.
    \end{equation*}
    \textbf{Se esiste}
    \begin{equation}\label{eq:hopital_ipotesi2}
        \lim_{x\rightarrow c^+}\frac{f'(x)}{g'(x)}
    \end{equation}
    \textbf{allora}
    \begin{equation}\label{eq:hopital_tesi}
        \lim_{x\rightarrow c^+}\frac{f(x)}{g(x)}=\lim_{x\rightarrow c^+}\frac{f'(x)}{g'(x)}.
    \end{equation}
\end{theorem}

\textbf{Il Teorema di de l'Hôpital vale anche se:}
\begin{itemize}
    \item il limite $\lim_{x\rightarrow c^+}$ in (\ref{eq:hopital_ipotesi})-(\ref{eq:hopital_tesi}) sostituito con $\lim_{x\rightarrow c^-},\,\lim_{x\rightarrow a^+}$ oppure $\lim_{x\rightarrow b^-}$,
    \item (\ref{eq:hopital_ipotesi}) e'
    \begin{equation*}
        \lim_{x\rightarrow c^+}f(x)=\pm\infty=\lim_{x\rightarrow c^+}g(x)
    \end{equation*}
\end{itemize}

\subsubsection{Uso del Teorema di del'Hôpital per lo studio della derivabilita'}
\begin{proposition}
    Sia $f$ continua in $[a,b]$ e derivabile in $(a,b)$. Se esiste $\lim_{x\rightarrow a^+}f'(x)$ allora
    \begin{equation*}
        f'(a):=\lim_{h\rightarrow 0^+}\frac{f(a+h)-f(a)}{h}=\lim_{x\rightarrow a^+}f'(x).
    \end{equation*}
\end{proposition}

\subsubsection{Applicazione Teorema di de l'Hopital}
\begin{example}
    \begin{equation*}
        \lim_{x\rightarrow+\infty}\frac{\log x}{x}=\left[\frac{\infty}{\infty}\right]\overset{H}{=}\frac{\frac{1}{x}}{1}=0.
    \end{equation*}
\end{example}

\subsection{Monotonia}

\begin{theorem}[Criterio di monotonia per funzioni derivabili]
    Sia $f$ continua in $[a,b]$ e derivabile in $(a,b)$. Allora
    \begin{equation*}
        \begin{matrix}
            f'(x)\geq 0 &\forall x\in(a,b)& \iff & f \text{ è crescente in } [a,b],\\
            f'(x)\leq 0 &\forall x\in(a,b)& \iff & f \text{ è decrescente in } [a,b].
        \end{matrix}
    \end{equation*}
\end{theorem}

Dato il Teorema di Fermat (Teorema \ref{th:teorema_fermat}) è possibile la seguente osservazione.

\begin{remark}\label{rem:minimo_relativo}
    Sia $f$ continua in $[a,b]$ e derivabile in $(a,b)$. Se
    \begin{itemize}
        \item $f'(x)<0$ per $x\in(a,x_0)$ allora $f$ è monotona decrescente su $(a,x_0)$,
        \item $f'(x_0)=0$,
        \item $f(x)>0$ per $x\in(x_0,b)$ allora $f$ è monotona crescente su $(x_0,b)$,
    \end{itemize}
    allora $x_0$ è un minimo relativo. Viceversa se $x_0$ è punto di massimo relativo.
\end{remark}

Studiare quando $f'(x_0)<0$ o $f'(x_0)>0$ è detto studio del segno della derivata prima di $f$. Questo determina se $x_0$ è effettivamente un minimo o un massimo relativo della funzione.

Riassumendo, per studiare i punti di massimo e minimo è necessario:
\begin{itemize}
    \item Calcolare $y = f'(x)$.
    \item Risolvere l'equazione $f'(x) = 0$ per trovare i candidati al ruolo di punti estremanti.
    \item Risolvere la disequazione $f'(x) > 0$ per conoscere il segno della derivata prima.
\end{itemize}

\begin{example}
    Sia
    \begin{equation*}
        f(x)=(x+1)\sqrt{\frac{x}{x+2}}\colon(-\infty,-2)\cup[0,+\infty)\rightarrow\mathbb R.
    \end{equation*}
    Allora,
    \begin{equation*}
        f'(x)=\frac{x^2+3x+1}{\sqrt{\frac{x}{x+2}}(x+2)^2}.
    \end{equation*}
    I punti stazionari sono $x_{1,2}=\frac{-3\pm\sqrt{3^2-4}}{2}=\frac{-3\pm\sqrt{5}}{2}$ (la $x$ con il + non appartiene al dominio di $f$). Dato che c'è l'asintoto orizzontale
    \begin{equation*}
        \lim_{x\rightarrow -2}f(x)=-\infty,
    \end{equation*}
    è possibile affermare che $\frac{-3-\sqrt{5}}{2}$ è un punto di massimo.
\end{example}


\subsection{Derivate Successive}
Una funzione $f$ derivabile $k$ volte in $[a,b]$ è indicata con
\begin{equation*}
    f^{(k)}(x)=\frac{d^k}{dx^k}f(x).
\end{equation*}

\begin{definition}[Insieme delle funzioni continua]
    Dato $I$ intervallo, $C(I)$ è l'insieme delle funzioni continue su $I$.
\end{definition}

\begin{definition}[Insieme delle funzioni derivabili]
    Dato $I$ intervallo, \gls{Ck(I)} è l'insieme delle funzioni derivabili $k$ volte su $I$.
\end{definition}

\subsection{Estremi locali}\label{ssec:estremi_locali_derivate}
Dal solo Teorema di Fermat (Teorema \ref{th:teorema_fermat}) non è possibile capire se $x_0$ è un estremo locale. Per scoprire se un \gls{punto stazionario} è un estremo sono utili i seguenti Teoremi. È supposto che che $f$ sia derivabile almeno due volte in $x_0$.

\begin{theorem}[Condizione necessaria per un estremo locale]
    Sia $x_0$ un punto di minimo locale interno al dominio di $f$. Allora
    \begin{equation*}
        f'(x_0)=0\quad\text{ e }f''(x_0)\geq 0.
    \end{equation*}
    $x_0$ è un punto di minimo locale interno al dominio di $f$ se
    \begin{equation*}
        f'(x_0)=0\quad\text{ e }f''(x_0)\leq 0.
    \end{equation*}
\end{theorem}

\begin{theorem}[Condizione sufficiente per un estremo locale]
    Sia $x_0$ un punto di minimo locale interno al dominio di $f$. Allora
    \begin{equation*}
        f'(x_0)=0\quad\text{ e }f''(x) > 0.
    \end{equation*}
    $x_0$ è un punto di minimo locale interno al dominio di $f$ se
    \begin{equation*}
        f'(x_0)=0\quad\text{ e }f''(x) < 0.
    \end{equation*}
\end{theorem}

\begin{theorem}[Condizione sufficiente per un estremo locale]
    Sia $x_0$ un punto di minimo locale interno al dominio di $f$. Se
    \begin{equation*}
        f'(x_0)=f''(x_0)=\hdots=f^{(k-1)}(x_0)=0 \text{ e } f^{(k)}(x_0)\neq 0,
    \end{equation*}
    allora
    \begin{itemize}
        \item se $k$ è pari e $f^{(k)}(x_0)>0$, $x_0$ è un punto di minimo locale,
        \item se $k$ è pari e $f^{(k)}(x_0)<0$, $x_0$ è un punto di massimo locale,
        \item se $k$ è dispari e $x_0$ non è un punto di estremo locale,
    \end{itemize}
\end{theorem}