\section{Successioni Numeriche}
\begin{definition}[Successione]
    Una successione e' una funzione indicata come
    \begin{equation*}
        \{a_n\}_{n\in\mathbb N}\colon\mathbb N\rightarrow\mathbb R,
    \end{equation*}
    dove $a_n$ e' il numero reale associato all'intero $n$.
\end{definition}

\begin{remark}
    Per una successione $\{a_n\}_{n\in\mathbb N}$ sono valide le Definizioni \ref{def:funzione_crescente}-\ref{def:funzione_monotona} di funzione crescente, strettamente crescente e monotona.
\end{remark}

Siano definite le seguenti successioni, per $n\in\mathbb N$:
\begin{equation*}
    \begin{matrix}
        \Tilde{a}_n=\frac{1}{n} & b_n = n^2 & c_n = \frac{1-(-1)^n}{2}\\
        &&\\
        d_n=|\{d\in\mathbb R\colon d|n\}| & e_n=\sum_{k=1}^n\frac{3}{10^k} & f_n=\frac{n}{2}\sin\frac{2\pi}{n}\\
        &&\\
        g_n= \frac{n}{n+1} & h_n=n! & i_n= (-1)^n\\
        &&\\
        j_n=\frac{(-1)^n}{n} &  & l_n = (-1)^n n
    \end{matrix}
\end{equation*}

\begin{remark}
    $b_n,\, e_n,\, f_n,\, g_n,$ e $h_n$ sono strettamente crescenti. $\Tilde{a}_n$ e' strettamente descrente. $c_n\, d_n$ e $j_n$ non sono monotone.
\end{remark}

\begin{definition}[Successione limitata]
    Una successione $\{a_n\}_{n\in\mathbb N}$ e' limitata se esistono $m$ e $M$ tali che
    \begin{equation*}
        m\leq a_n\leq M,\quad \forall n\in\mathbb N.
    \end{equation*}
\end{definition}

Cosi' anche se esiste $M>0$ tale che
\begin{equation*}
    |a_n|\leq M,\quad \forall n\in\mathbb N.
\end{equation*}

\begin{definition}[Sottosucessione]\label{def:sottosuccessione}
    Sia $\{a_n\}$ una successione. Diciamo che $\{a_{n_k}\}$ e' una
    sottosuccessione di $\{a_n\}$ se la successione di numeri naturali
    \begin{equation*}
        \begin{aligned}
            \mathbb N &\rightarrow\mathbb N\\
            k &\mapsto n_k
        \end{aligned}
    \end{equation*}
    e' strettamente crescente
\end{definition}

\subsection{Successioni convergenti}
\begin{definition}[Successione convergente]
    Sia $L\in\mathbb R$. La successione $\{a_n\}_{n\in\mathbb N}$ converge a $L$, ovvero
    \begin{equation*}
        \lim_{n\rightarrow+\infty}a_n=L,
    \end{equation*}
    se per ogni $\varepsilon>0$ esiste $n_\varepsilon$ tale che
    \begin{equation}\label{eq:accuratezza_limite_successione_convergente}
        |a_n-L|<\varepsilon,\,\forall n>n_\varepsilon.
    \end{equation}
\end{definition}

\begin{remark}\label{rem:successione_convergente_limitata}
    La condizione (\ref{eq:accuratezza_limite_successione_convergente}) equivale a
    \begin{equation*}
        L-\varepsilon < a_n < L+\varepsilon,
    \end{equation*}
    ovvero
    \begin{equation*}
        a_n\in(L-\varepsilon,L+\varepsilon).
    \end{equation*}
\end{remark}

\begin{definition}[Successione infinitesima]
    Una successione è detta infinitesima se tende a 0.
\end{definition}

\begin{proposition}
    La successione $a_n$ è infinitesima se e solo se $|a_n|$ è infinitesima.
\end{proposition}
\begin{proof}
    Pagina 35.
\end{proof}

\begin{example}       $\lim_{n\rightarrow+\infty}\Tilde{a} = 0$
\end{example}

\begin{example}
    $\lim_{n\rightarrow+\infty}j_n = 0.$
\end{example}

\begin{example}
    $\lim_{n\rightarrow+\infty}e_n = \frac{1}{3}.$
\end{example}

\begin{example}
    $\lim_{n\rightarrow+\infty}g_n = 1.$
\end{example}

\begin{theorem}[Unicita' del limite]
    Una successione convergente non puo' avere due limiti distinti.
\end{theorem}
\begin{proof}
    Vedere pagina 29.
\end{proof}

\begin{example}[Successione non convegente]
    La successione $c_n$ vale alternativamente 0 oppure 1, quindi non potrà essere contenuta in un raggio minore di $\frac{1}{2}$. Lo stesso $i_n$.
\end{example}

\begin{theorem}\label{th:successione_convergente_limitata}
    Ogni successione convergente e' limitata.
\end{theorem}
\begin{proof}
    Vedere pagina 30.
\end{proof}

\begin{remark}
    Il Teorema \ref{th:successione_convergente_limitata} afferma che la "limitazione" e' una condizione necessaria ma non sufficiente, quindi non afferma che una successione limitata e' convergente.
\end{remark}

\begin{example}[Successioni non convergenti]
    Le successioni $b_n,\, d_n$ e $h_n$ non sono limitate quindi non possono essere convergenti. Le successioni $c_n$ ed $i_n$ sono limitate ma non convergenti.
\end{example}


\begin{proposition}
    Data ${a_n}_{n\in\mathbb N}$, se e' convergente allora ha le seguenti proprieta':
    \begin{enumerate}
        \item \textbf{Unicita':} ${a_n}_{n\in\mathbb N}$ ha al piu' un limite,
        \item \textbf{Limitatezza:} Osservazione \ref{rem:successione_convergente_limitata},
        \item \textbf{Sottinsieme convergente:} Ogni sottoinsieme di ${a_n}_{n\in\mathbb N}$ converge allo stesso limite di ${a_n}_{n\in\mathbb N}$.
    \end{enumerate}
\end{proposition}

\subsubsection{Operazioni con i limiti}

\begin{proposition}
    Se $\lim_{n\rightarrow+\infty}a_n=a$ e $\lim_{n\rightarrow+\infty}b_n=b$, per ogni $a,b\in\mathbb R$ allora:
    \begin{itemize}
        \item $\lim_{n\rightarrow+\infty}(\alpha a_n+\beta b_n)=\alpha a_n+\beta b_n,\quad \forall\alpha ,\,\beta\in\mathbb R,$
        \item $\lim_{n\rightarrow+\infty}(a_n\cdot b_n)=a\cdot b,$
        \item $\lim_{n\rightarrow+\infty}\frac{a_n}{b_n}=\frac{a}{b}\quad \text{ se } b\neq 0.$
    \end{itemize}
\end{proposition}

\begin{proposition}
    Il prodotto tra una successione infinitesima e limitata è una successione infinitesima.
\end{proposition}

\subsection{Successioni divergenti ed indeterminate}
\begin{definition}\label{def:successione_divergente}[Successione divergente]
    La successione $\{a_n\}_{n\in\mathbb N}$ diverge a $+\infty$, ovvero
    \begin{equation*}
        \lim_{n\rightarrow+\infty}a_n=+\infty,
    \end{equation*}
    se per ogni $M>0$ esiste $n_M$ tale che $a_n>M$ per ogni $n>n_M$.\\
    La successione $\{a_n\}_{n\in\mathbb N}$ diverge a $-\infty$, ovvero
    \begin{equation*}
        \lim_{n\rightarrow+\infty}a_n=-\infty,
    \end{equation*}
    se per ogni $M>0$ esiste $n_M$ tale che $a_n<-M$ per ogni $n>n_M$.    
\end{definition}

$n_M$ è da considerare come numero naturale oltre il quale ogni valore assunto dalla successione è maggiore di $M$.

\begin{example}[Successioni divergenti]
    $b_n$ e $h_n$ divergono a $+\infty$.
\end{example}

\paragraph{N.B.:}\textbf{Una successione ha limite se e' convergente oppure divergente.}

\begin{definition}[Successione indeterminata]
    Una successione che non e' divergente o convergente e' detta indeterminata (non ha limite).
\end{definition}

\begin{example}[Successioni indeterminate]
    $d_n$ ed $l_n$ sono indeterminate. $l_n$ e' illimitata superiormente ed inferiormente, non diverge perche' assume infinite volte, in modo alternato, valori positivi e negativi.
\end{example}

\begin{remark}\label{re:successione_divergente_illimitata}
    Dalla Definizione \ref{def:successione_divergente} di Successione divergente e' possibile osservare che ogni successione divergente e' illimitata.
\end{remark}

\subsubsection{Operazioni con i limiti}
Per le operazioni tra successioni divergenti valgono le seguenti proprieta'.
\begin{property}[Somma di successioni]
    Vedere Tabella \ref{tab:somma_successioni_divergenti}.
    \begin{table}[!hbt]
        \centering
        \begin{tabular}{|c|c|c|}
            \hline
            $a_n\rightarrow a\in\mathbb R$ & $b_n\rightarrow\pm\infty$ & $a_n+b_n = \pm\infty$\\
            \hline
            $a_n \rightarrow\pm\infty$ & $b_n\rightarrow\pm\infty$ & $a_n+b_n\rightarrow\pm\infty$\\
            \hline
        \end{tabular}
        \caption{Somma di successioni divergenti}
        \label{tab:somma_successioni_divergenti}
    \end{table}
\end{property}

\begin{property}[Prodotto di successioni]
    Vedere Tabella \ref{tab:prodotto_successioni_divergenti}.
    \begin{table}[!hbt]
        \centering
        \begin{tabular}{|c|c|c|}
            \hline
            $a_n\rightarrow a\neq 0$ & $b_n\rightarrow\pm\infty$ & $a_nb_n = \pm(sgn(a))\infty$\\
            \hline
            $a_n \rightarrow\pm\infty$ & $b_n\rightarrow\pm\infty$ & $a_n+b_n\rightarrow+\infty$\\
            \hline
            $a_n \rightarrow\mp\infty$ & $b_n\rightarrow\pm\infty$ & $a_n+b_n\rightarrow-\infty$\\
            \hline
        \end{tabular}
        \caption{Prodotto di successioni divergenti}
        \label{tab:prodotto_successioni_divergenti}
    \end{table}
\end{property}

\begin{property}[Rapporto di successioni]
    Vedere Tabella \ref{tab:rapporto_successioni_divergenti}.
    \begin{table}[!hbt]
        \centering
        \begin{tabular}{|c|c|c|}
            \hline
            $a_n\rightarrow a\in\mathbb R$ & $b_n\rightarrow\pm\infty$ & $\frac{a_n}{b_n}=0$\\
            \hline
            $a_n \rightarrow\pm\infty$ & $b_n\rightarrow b\neq 0$ & $\frac{a_n}{b_n}\rightarrow \pm(sgn(b))\infty$\\
            \hline
        \end{tabular}
        \caption{Rapporto di successioni divergenti}
        \label{tab:rapporto_successioni_divergenti}
    \end{table}
\end{property}

\begin{property}[Rapporto di Successioni con Successioni infinitesime]
    Vedere Tabella \ref{tab:proprieta_successioni_divergenti}.
    \begin{table}[!hbt]
        \centering
        \begin{tabular}{|c|c|c|}
            \hline
            $a_n\rightarrow a\neq 0$ & $b_n\rightarrow 0$ & $\frac{a_n}{b_n} = +\infty$\\
            \hline
            $a_n \rightarrow\pm\infty$ & $b_n\rightarrow 0$ & $\frac{a_n}{b_n} = +\infty$\\
            \hline
        \end{tabular}
        \caption{Somma di Successioni divergenti}
        \label{tab:proprieta_successioni_divergenti}
    \end{table}
\end{property}

\begin{remark}
    Anche nel caso della Tabella \ref{tab:proprieta_successioni_divergenti} la successione $\frac{a_n}{b_n}$ potrebbe non avere limite. Vedere il caso $a_n=1,\, b_n=\frac{(-1)^n}{n}$.
\end{remark}

\subsection{Confronto fra successioni}
\begin{theorem}[Teorema della permanenza del segno]
    La successione $a_n$ è definitivamente positiva se
    \begin{equation*}
        \lim_{n\rightarrow +\infty}a_n=a>0.
    \end{equation*}
\end{theorem}
\begin{proof}
    Pagina 32.
\end{proof}

Se la successione converge ad un numero negativo allora è definitivamente negativa.

\begin{corollary}
    Se $a_n\geq 0,\,\forall n\in\mathbb N$ e $\lim_{n\rightarrow +\infty}a_n=a$, allora $a\geq 0$.
\end{corollary}

\begin{remark}
    Se la successione $a_n$ è sempre positiva, non è possibile concludere che il limite è strettamente positivo. Un esempio è $a_n=\frac{1}{n}$.
\end{remark}

\begin{corollary}[Confronto per successioni convergenti]
    Se $a_n\leq b_n,\,\forall n\in\mathbb N$ e $\lim_{n\rightarrow\infty}a_n=a,\, \lim_{n\rightarrow\infty}b_n=b$, allora $a\leq b$.
\end{corollary}

\begin{theorem}[Teorema dei due carabinieri]
    Siano $a_n,b_n,c_n$ successioni tali che
    \begin{equation*}
        a_n\leq b_n\leq c_n,\,\forall n\in\mathbb N.
    \end{equation*}
    Se
    \begin{equation*}
        \lim_{n\rightarrow+\infty}a_n=\lim_{n\rightarrow+\infty}c_n=P\in\mathbb R,
    \end{equation*}
    allora
    \begin{equation*}
        \lim_{n\rightarrow+\infty}c_n=P.
    \end{equation*}
\end{theorem}
\begin{proof}
    Pagina 33.
\end{proof}

\begin{remark}
    Le successioni $a_n$ e $c_n$ devono convergere allo stesso limite.
\end{remark}

\begin{example}
    Da
    \begin{equation*}
        0\leq \frac{n^2}{n^3+1}\leq\frac{1}{n}
    \end{equation*}
    segue che
    \begin{equation*}
        \lim_{n\rightarrow+\infty}\frac{n^2}{n^3+1}=0.
    \end{equation*}
\end{example}

\subsection{Forme Indeterminate}
\begin{definition}[Forme inderminate]
    \begin{equation*}
        [\infty - \infty],\quad [0\cdot\infty],\quad\left[\frac{0}{0}\right],\quad\left[\frac{\infty}{\infty}\right].
    \end{equation*}
\end{definition}

\subsection{Alcuni limiti notevoli}
\begin{proposition}
    \begin{equation*}
        \lim_{n\rightarrow+\infty}n^\alpha=
        \begin{cases}
            +\infty & \text{se } \alpha>0,\\
            1 &\text{se } \alpha = 0,\\
            0 &\text{se } \alpha<0.
        \end{cases}
    \end{equation*}
\end{proposition}

\begin{proposition}
    Se $b>0$ e $\lim_{n\rightarrow+\infty}a_n=a\in\mathbb R$, allora
    \begin{equation*}
        \lim_{n\rightarrow+\infty} b^{a_n}=b^a.
    \end{equation*}
    Se $\lim_{n\rightarrow+\infty}a_n =\infty$, allora
    \begin{equation*}
        \lim_{n\rightarrow+\infty}b^a_n=
        \begin{cases}
            +\infty & \text{se } b>1,\\
            1 &\text{se } b = 1,\\
            0 &\text{se } 0<b<1.
        \end{cases}
    \end{equation*}
\end{proposition}

\begin{proposition}\label{prop:successione_positiva_crescente}
    Se $\alpha\in\mathbb R$ e $a_n$ è una successione positiva convergente ad $a>0$, allora
    \begin{equation*}
        \lim_{n\rightarrow+\infty}a_n^\alpha =a^\alpha.
    \end{equation*}
    Se $\lim_{n\rightarrow+\infty}a_n=+\infty$, allora
    \begin{equation*}
        \lim_{n\rightarrow+\infty}a_n^\alpha=
        \begin{cases}
            +\infty & \text{se } \alpha>0,\\
            1 &\text{se } \alpha = 0,\\
            0 &\text{se } \alpha<0.
        \end{cases}
    \end{equation*}
\end{proposition}

\begin{proposition}\label{prop:limite_log}
    Se $b>0,\, b\neq 1,\, \lim_{n\rightarrow+\infty}a_n=a$ con $a_n,\, a>0$, allora
    \begin{equation*}
        \lim_{n\rightarrow+\infty}\log_b a_n=\log_b a.
    \end{equation*}
    Se $\lim_{n\rightarrow+\infty}a_n=+\infty$ e $b>1$ allora
    \begin{equation*}
        \lim_{n\rightarrow+\infty}\log_b a_n=+\infty,
    \end{equation*}
    mentre $\lim_{n\rightarrow+\infty}a_n=0$ e $b>1$
    \begin{equation*}
        \lim_{n\rightarrow+\infty}\log_b a_n=-\infty.
    \end{equation*}
\end{proposition}

\subsection{Successioni monotone}
\begin{theorem}
    Ogni successione monotona $\{a_n\}_{n\in\mathbb N}$ ha un limite.\\
    Se $\{a_n\}_{n\in\mathbb N}$ è crecente, allora
    \begin{equation*}
        \lim_{n\rightarrow+\infty} a_n = \sup\{a_n\colon n\in\mathbb N\},
    \end{equation*}
    se decrescente, allora
    \begin{equation*}
        \lim_{n\rightarrow+\infty} a_n = \inf\{a_n\colon n\in\mathbb N\}.
    \end{equation*}
\end{theorem}

\begin{corollary}
    Ogni successione monotona e limitata converge.
\end{corollary}

\subsection{Criterio del rapporto e confronto fra infiniti}
\begin{theorem}[Criterio del rapporto]
    Sia $\{a_n\}_{n\in\mathbb N}$ una successione a termini positivi tali che esiste
    \begin{equation*}
        \lambda=\lim_{n\rightarrow+\infty}\frac{a_{n+1}}{a_n}.
    \end{equation*}
    Allora, se $\lambda<1$,
    \begin{equation*}
        \lim_{n\rightarrow+\infty}a_n=0,
    \end{equation*}
    oppure
    \begin{equation*}
        \lim_{n\rightarrow+\infty}a_n=\infty.
    \end{equation*}
\end{theorem}

Il teorema afferma che per $\alpha>0$ e $a>1$, $n^\alpha,\, a^n,\, n!$ sono infiniti e sono elencati in ordine crescente. Il confronto tra due successioni che divergono all'infinito si esprime calcolando il limite del rapporto tra le due successioni.

\begin{example}
    Per mostrare che
    \begin{equation*}
        \lim_{n\rightarrow+\infty}\frac{a^n}{n!}=0,
    \end{equation*}
    è osservato che
    \begin{equation*}
        \lim_{n\rightarrow+\infty}\frac{a^{n+1}}{(n+1)!}\,\frac{n!}{a^n}=\lim_{n\rightarrow+\infty}\frac{a}{(n+1)}=0=\alpha<1.
    \end{equation*}
\end{example}

\subsection{Numero di Nepero e limiti ad esso associati}
\begin{theorem}
    La successione $a_n=\left(1+\frac{1}{n}\right)^n$ converge ad $e$, ovvero:
    \begin{equation*}
        e = \lim_{n\rightarrow+\infty}\left(1+\frac{1}{n}\right)^n.
    \end{equation*}
\end{theorem}
\begin{proof}
    Pagina 41.
\end{proof}

\begin{corollary}
    Sia $\{a_n\}_{n\in\mathbb N}$ una successione divergente, allora
    \begin{equation*}
        \lim_{n\rightarrow+\infty}\left(1+\frac{1}{a_n}\right)^{a_n}=e.
    \end{equation*}
\end{corollary}

\begin{remark}[Limite notevole]
    Tramite la Proposizione \ref{prop:successione_positiva_crescente}:
    \begin{equation*}
        \lim_{n\rightarrow+\infty}\left(1+\frac{x}{n}\right)^{n}=e^x.
    \end{equation*}
    Per $x=0$ è un limite facile, per $x\neq 0$ è possibile osservare che
    \begin{equation*}
        \left(1+\frac{x}{n}\right)^n=\left[\left(1+\frac{1}{\frac{n}{x}}\right)^{\frac{n}{x}}\right]^x
    \end{equation*}
    e quindi, dato che che $\frac{n}{x}$ diverge a $sgn(x)\infty$
    \begin{equation*}
        \lim_{n\rightarrow+\infty}\left(1+\frac{1}{\frac{n}{x}}\right)^{\frac{n}{x}}=e.
    \end{equation*}
\end{remark}

\begin{remark}[Limite notevole]
    Dalla Proposizione \ref{prop:limite_log} segue, se $\lim_{n\rightarrow+\infty}a_n=0$ allora
    \begin{equation}\label{eq:limite_notevole_log}
        \boldsymbol{\lim_{n\rightarrow+\infty}\frac{\log(1+a_n)}{a_n}}\overset{\footnotemark}{=}\lim_{n\rightarrow+\infty}\log\left((1+a_n)^{\frac{1}{a_n}}\right)=\log e=\boldsymbol 1.
    \end{equation}
    La successione $\frac{1}{a_n}$ diverge solo se $a_n$ ha segno definitivamente costante. Se questo non vale è necessario considerare la parte negativa e positiva separatamente affichè il risultato rimanga valido.
\end{remark}

\footnotetext{Applicazione $y\log x=\log x^y$.}

\begin{remark}[Limite notevole]
    Cambiando la base di (\ref{eq:limite_notevole_log}), il limite notevole diventa
    \begin{equation*}        \lim_{n\rightarrow+\infty}\frac{\log_a(1+a_n)}{a_n}=\log_a e.
    \end{equation*}
\end{remark}

\begin{remark}
    Se $\lim_{n\rightarrow+\infty}a_n=0$, allora
    \begin{equation*}
        \lim_{n\rightarrow+\infty}\frac{e^{a_n}-1}{a_n}=1
    \end{equation*}
\end{remark}
\begin{proof}
    Sia $b_n=e^{a_n}-1$ e $\lim_{n\rightarrow+\infty}b_n=0$, allora
    \begin{equation*}
        \lim_{n\rightarrow+\infty}\frac{e^{a_n}-1}{a_n}=\lim_{n\rightarrow+\infty}\frac{b_n}{\log(1+b_n)}=1
    \end{equation*}
\end{proof}

\subsection{Confronto fra infiniti}
\begin{theorem}
    Per $a,\,b>1$ e $\alpha>0$ vale
    \begin{equation*}
        \lim_{n\rightarrow+\infty}\frac{\log_b n}{n^\alpha}=\lim_{n\rightarrow+\infty}\frac{n^\alpha}{a^n}=\lim_{n\rightarrow+\infty}\frac{a^n}{n!}=\lim_{n\rightarrow+\infty}\frac{n!}{n^n}=0,
    \end{equation*}
    ovvero
    \begin{equation*}
        \log_b n<n^\alpha<a^n<n!<n^n.
    \end{equation*}
\end{theorem}
\begin{proof}
    Pagina 44.
\end{proof}